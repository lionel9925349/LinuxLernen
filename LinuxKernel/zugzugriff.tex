Mit Pfaden arbeiten
• Der Befehl cd nimmt Pfadangaben als Parameter entgegen. Diese können wir absolut oder
relativ angeben
o Absolute Pfade
Starten mit einem / und beschreiben den Pfad zum gewünschten Verzeichnis
ausgehend vom obersten Punkt im Dateisystem
o Relative Pfade
Beschreiben den Weg zum Ziel ausgehend vom aktuellen Standort
• Die Variable $OLDPWD enthält das vorige Verzeichnis, indem man sich befunden hat
• Um ein Programm auszuführen, haben wir zwei Möglichkeiten:
o Wir geben den absoluten Pfad vorneweg mit an
o Wir schreiben ./ davor, das steht für das aktuelle Verzeichnis
• Wenn wir Dateien öffnen wollen, zBsp. mit einem Editor, können wir direkt die Datei
angeben, wenn sie sich im selben Verzeichnis befindet
Verzeichnisse erstellen und löschen
• Unter Ubuntu wird in der Datei .profile im Homeverzeichnis festgelegt, welche Pfade in
der $PATH Variable aufgenommen werden
• Unter CentOS wird in der Datei .bash_profile im Homeverzeichnis festgelegt welche Pfade
in der $PATH Variable aufgenommen werden
• Mit mkdir werden Verzeichnisse erstellt, mit rmdir gelöscht (nur leere Verzeichnisse ohne
Unterordner). Mit der Option -p auch mit Unterordner
• Der Befehl rm löscht sowohl Dateien als auch Verzeichnisse. 




Verzeichnis-Listings verstehen
• Ausgabe von des Befehls ls -l (langes Format)
o Jede Datei bzw. jedes Unterverzeichnis stehen in einer eigenen Zeile
o Der Punkt steht für das aktuelle Verzeichnis und der Doppelpunkt für das
übergeordnete Verzeichnis
o In der ersten Spalte steht der Dateityp
▪ d steht für Directory, also ein Verzeichnis
▪ – steht für normale Dateien
▪ l wie Link, das sind Symbolische oder Softlinks
▪ c für Character Device, das sind zeichenorientierte Geräte (Modems)
▪ b für Block Device, das sind blockorientierte Geräte (Speichermedien)
o Ausführbare Dateien werden in grün dargestellt (anhand der Rechte)
o Die Rechte werden entsprechend angezeigt
o Die Anzahl der Links auf diesen Eintrag
o Der Eigentümer des Eintrags und die Gruppe, der der Eintrag zugewiesen wurde
o Die Größe der Datei wird in Bytes angegeben
o Datum der letzten Änderung
• Der Befehl ls -F kann durch angehängt Zeichen kennzeichnen, um welche Art von Eintrag es
sich handelt:
o / Verzeichnisse
o * ausführbare Dateien
o @ symbolische Links
• Der Befehl ls unterstützt sehr viele Optionen die flexibel miteinander kombiniert werden
können. Siehe dazu Man-Page von ls und Befehlsübersicht dieses Kurses
• Das Programm tree zeigt sehr übersichtlich den Verzeichnisbaum ab einem bestimmten
Punkt an