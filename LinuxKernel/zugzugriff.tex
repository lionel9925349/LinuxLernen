Mit Pfaden arbeiten
• Der Befehl cd nimmt Pfadangaben als Parameter entgegen. Diese können wir absolut oder
relativ angeben
o Absolute Pfade
Starten mit einem / und beschreiben den Pfad zum gewünschten Verzeichnis
ausgehend vom obersten Punkt im Dateisystem
o Relative Pfade
Beschreiben den Weg zum Ziel ausgehend vom aktuellen Standort
• Die Variable $OLDPWD enthält das vorige Verzeichnis, indem man sich befunden hat
• Um ein Programm auszuführen, haben wir zwei Möglichkeiten:
o Wir geben den absoluten Pfad vorneweg mit an
o Wir schreiben ./ davor, das steht für das aktuelle Verzeichnis
• Wenn wir Dateien öffnen wollen, zBsp. mit einem Editor, können wir direkt die Datei
angeben, wenn sie sich im selben Verzeichnis befindet
Verzeichnisse erstellen und löschen
• Unter Ubuntu wird in der Datei .profile im Homeverzeichnis festgelegt, welche Pfade in
der $PATH Variable aufgenommen werden
• Unter CentOS wird in der Datei .bash_profile im Homeverzeichnis festgelegt welche Pfade
in der $PATH Variable aufgenommen werden
• Mit mkdir werden Verzeichnisse erstellt, mit rmdir gelöscht (nur leere Verzeichnisse ohne
Unterordner). Mit der Option -p auch mit Unterordner
• Der Befehl rm löscht sowohl Dateien als auch Verzeichnisse. 