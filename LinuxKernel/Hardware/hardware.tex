Die Kernel-Verzeichnisse /proc/ und /sys/
• Es gibt zahlreiche virtuelle Dateisysteme für die verschiedenen Zwecke
o Das Dateisystem proc, wird unter /proc gemountet
o Das Dateisystem sysfs wird unter /sys gemountet
• /sys und /proc stellen virtuelle Kernel-Verzeichnisse bereit
• Sie existieren nur im Arbeitsspeicher und werden vom Kernel dynamisch erstellt
• Die Daten darin werden beim Herunterfahren nicht gespeichert
• Wichtige Dateien in /proc
o cpuinfo: Zeigt z.B. Informationen zum Prozessor
o interrupts: Zeigt Anfragen für Datenaustausch mit dem Prozessor
o ioports: Zeigt die Input-Output-Adressen der Hardware-Komponenten. Über sie ist
ein Datenaustausch zwischen der CPU und dem jeweiligen Gerät möglich
o meminfo: Zeigt die Verwendung des Arbeitsspeichers
o swaps: Hier wird der verwendete Swap-Space aufgeführt
o version: Zeigt die Linux-Version des Systems an
o Unter /proc befindet sich das Unterverzeichnis /sys mit weiteren
Unterverzeichnissen für verschiedene Aspekte des Systems. Das umfasst Hardware,
wie dev, fs oder net, aber auch Software, wie z.B. users
o Die Unterverzeichnisse mit den Ziffern entsprechen den Prozess-IDs und beinhalten
Dateien zu diesem jeweiligen Prozess
• Die Dateien unter /proc können auch zur Manipulation und zum Tuning des Systems
angepasst werden
o Beispiel: Unter /proc/sys/net/ipv4 diverse Tuning-Parameter für den TCP/IP-Stack
unter IPv4
• Im Verzeichnis /sys befinden sich wie unter /proc Informa













Das Geräteverzeichnis /dev unter der Lupe
• Im Geräteverzeichnis /dev befinden sich aktive Komponenten, und auch Geräte die nicht
permanent aktiv sind:
o Character Devices, die nur seriell angesprochen werden können, also bit für bit
hintereinander lesend und schreibend.
Zum Beispiel die virtuellen Terminals tty xy
o Block Devices, auf die wahlfrei und blockweise zugegriffen wird. Das bedeutet, dass
nicht erst die ersten 1000 Bytes des Datenträgers ausgelesen werden müssen, bevor
die Daten ab dem 1001. Byte gelesen werden können. Stattdessen kann gezielt auf
die gewünschten Speicherblöcke zugegriffen werden.
Zum Beispiel Festplatten und andere Speichergeräte
o Die drei Ein- und Ausgabekanäle Stdin, stdout und stderr verweisen auf
/proc/self/fd/
o cdrom und dvd zeigen auf sr0. Dies ist die Gerätedatei für optische Laufwerke
o Hinter /dev/random steckt ein hochwertigen Zufallsgenerator
o Hinter /dev/urandom steckt ein nicht so hochwertigen Zufallsgenerator
o /dev/zero liefert die geforderte Anzahl an Null-Bytes zurück
o Bei /dev/null handelt es sich um den virtuellen Mülleimer. Alles was hier hin
geschrieben wird ist unwiederbringlich verloren
• Das Verzeichnis /dev ist Mountpoint für eine virtuelle Komponente namens udev und das
verwendete Pseudo-Dateisystem nennt sich devtmpfs. Dieses Verzeichnis wird dynamisch
vom Kernel und seinen Komponenten gefüllt und die hier aufgeführten Dateien sind nur
virtuell vorhanden
• Auch die Unterverzeichnisse unter /dev sind dynamisch erzeugte Objekte, die zur besseren
Verwaltung dienen und meistens Symlinks bereitstellen. Denn wenn sich die
Gerätebezeichnungen ändern, weil z.B. Festplatten intern umgesteckt werden, so werden
hier einfach die Symlinks korrigiert und es passt wieder 