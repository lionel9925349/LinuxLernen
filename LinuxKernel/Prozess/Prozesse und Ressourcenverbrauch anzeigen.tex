Prozesse und Ressourcenverbrauch anzeigen
 Der Befehl ps ist vielfältig und kann in unterschiedlicher Notation genutzt werden:
o Linux-Notation: Optionen werden mit Minus (-) eingeleitet
o BSD-Notation: Optionen werden ohne Minus geschrieben
o GNU-Notation: Doppeltes Minus als Einleitung, danach komplette Wörter
 Insbesondere Linux- und BSD-Notation nutzen unterschiedliche Buchstaben für dieselben
Optionen, z.B. ps -e = ps ax

 Welche Bedeutung die einzelnen Spalten haben, zeigt man ps
 Der Befehl top zeigt die Prozesse und deren Ressourcennutzung dynamisch an
 Mit htop wird die Darstellung noch etwas schöner
 Beide Tools ermöglichen die Darstellung nach Spalten, htop unterstützt zudem Filter
 Mit pstree können die Prozesse hierarchisch angezeigt werden
Programme im Vordergrund und im Hingergrund ausführen und verwalten
 Programme sperren normalerweise das Terminal, von dem aus sie aufgerufen wurden
 Programme können mit STRG+C beendet werden
 Programme können mit & im Hintergrund gestartet werden, sodass das Terminal weiter
nutzbar bleibt
 Jobs können mit fg und bg in den Vordergrund und zurück verschoben werden
 Wird die Shell beendet, werden auch alle in dieser Shell gestarteten Programme beendet
 Mit nohup <Prozess> kann ein Prozess so gestartet werden, dass er das HUP-Signal ignoriert,
wenn die Shell beendet wird. Seine Ausgaben werden in nohup.out geschrieben
 Mit disown <BG-ID> kann ein Prozess nachträglich von der Shell getrennt werden