Prozesse und Ressourcenverbrauch anzeigen
 Der Befehl ps ist vielfältig und kann in unterschiedlicher Notation genutzt werden:
o Linux-Notation: Optionen werden mit Minus (-) eingeleitet
o BSD-Notation: Optionen werden ohne Minus geschrieben
o GNU-Notation: Doppeltes Minus als Einleitung, danach komplette Wörter
 Insbesondere Linux- und BSD-Notation nutzen unterschiedliche Buchstaben für dieselben
Optionen, z.B. ps -e = ps ax

 Welche Bedeutung die einzelnen Spalten haben, zeigt man ps
 Der Befehl top zeigt die Prozesse und deren Ressourcennutzung dynamisch an
 Mit htop wird die Darstellung noch etwas schöner
 Beide Tools ermöglichen die Darstellung nach Spalten, htop unterstützt zudem Filter
 Mit pstree können die Prozesse hierarchisch angezeigt werden
Programme im Vordergrund und im Hingergrund ausführen und verwalten
 Programme sperren normalerweise das Terminal, von dem aus sie aufgerufen wurden
 Programme können mit STRG+C beendet werden
 Programme können mit & im Hintergrund gestartet werden, sodass das Terminal weiter
nutzbar bleibt
 Jobs können mit fg und bg in den Vordergrund und zurück verschoben werden
 Wird die Shell beendet, werden auch alle in dieser Shell gestarteten Programme beendet
 Mit nohup <Prozess> kann ein Prozess so gestartet werden, dass er das HUP-Signal ignoriert,
wenn die Shell beendet wird. Seine Ausgaben werden in nohup.out geschrieben
 Mit disown <BG-ID> kann ein Prozess nachträglich von der Shell getrennt werden


Prozesse beenden
 Jeder Prozess hat eine automatisch vom Betriebssystem zugewiesene PID
 Mit dem Befehl kill können diverse Signale an die Prozesse gesendet werden:
o SIGTERM (15): normale Beendigung eines Prozesses
o SIGKILL (9): harte Beendigung des Prozesses
o SIGINT (2): Wird von STRG+C gesendet, wirkt wie SIGTERM
o SIGSTOP (19): Wird von STRG+Z gesendet, hält den Prozess an
o SIGHUP (1): Sendet das Terminal beim Beenden
 Mit killall <Zeichenkette> können alle Prozesse beendet werden, die der Zeichenkette
entsprechen
Kurs: LPIC-1 Linux-Bootcamp - In 30 Tagen zum Linux-Admin
Trainer: Eric Amberg & Jannis Seemann
© Eric Amberg und Jannis Seemann Seite 2
Prozess-Prioritäten mit Nice-Levels steuern
 Jedem Prozess wird vom Linux-System bzw. der CPU ein bestimmter Grad an
Aufmerksamkeit zugestanden
 Dieser Grad wird durch das Nice-Level bestimmt, Standard ist 0
 Die Priorität liegt zwischen -20 (höchste) und +19 (niedrigste)
 Prioritäten können mit dem Programm nice beim Start des Programms gesetzt werden
 Nur root darf Nice-Level heruntersetzen, Programme also höher priorisieren
 Ein Programm mit Nice-Level -20 kann das System beeinträchtigen, dies sollte nur sehr
vorsichtig eingesetzt werden
 Mit renice können die Prioritäten eines laufenden Prozesses angepasst werden
 Nice-Levels sind keine Booster-Funktion und können nur als eines von mehreren TuningMitteln gesehen werden