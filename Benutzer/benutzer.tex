Benutzerverwaltung
• Linux ist ein Multiuser-System: Es können mehrere Benutzer das System verwenden, auch
gleichzeitig
• Jeder Benutzer hat normalerweise sein eigenes Benutzer-Konto
• Obligatorisch ist ein Benutzername, mit der der Benutzer sich am System anmeldet. Dieser
Name muss auf dem System eindeutig sein
• Linux unterscheidet generell zwischen Groß- und Kleinschreibung, das gilt auch für die
Benutzernamen
• Die UID, also die User-ID muss ebenfalls eindeutig sein und wird normalerweise vom System
automatisch vergeben
o Spezielle Benutzer haben IDs unter 1000, Root z.B. hat immer die ID 0
o Normale Benutzer erhalten eine ID von 1000 aufwärts
• Benutzer erhalten eine Login-Shell. Das ist die Umgebung, die für sie bereitgestellt wird,
wenn sie sich auf der Konsole einloggen
• Dash steht für Debian Almquist Shell und wird als Standard-Shell für Debian-Derivate
verwendet
• Einem Benutzer wird ein Home-Verzeichnis zugewiesen
o root hat sein Home-Verzeichnis unter /root
o Normaler Benutzer in der Regel unter /home/Benutzername
• Einstellungen im Zusammenhang mit useradd und userdel werden in der Datei
/etc/login.defs festgelegt
• In der Datei /etc/default/useradd können weitere Voreinstellungen getroffen werden
(zBsp. Festlegung der Login-Shell als Standard bei der Erstellung eines neuen Benutzers)
• Gruppen werden verwendet, um mehreren Accounts Zugriffsrechte auf eine Ressource
zuzuweisen
• Jedem Benutzer wird eine Hauptgruppe zugewiesen. Standardmäßig wird hierfür eine
Gruppe mit demselben Namen und derselben ID wie die des Benutzers angelegt
• Als Alternative erhalten alle Benutzer dieselbe Hauptgruppe namens users. Sie hat meistens
die ID 100. Das ist jedoch nicht so flexibel, daher erhalten Benutzer auf den meisten
Systemen per Default ihre eigene Gruppe
• Benutzer können Mitglied mehrerer Gruppen sein und werden teilweise automatisch
bestimmten zusätzlichen Gruppen zugewiesen

-Der Befehl usermod unterstützt fast alle Optionen, die auch useradd bereitstellt. Leider
funktioniert die Option –m nicht wie bei useradd . Daher müssen das Home-Verzeichnis
manuell kopiert, und die Zugriffsrechte entsprechend gesetzt werden
• Es existieren diverse Systembenutzer. Denn aus Sicherheitsgründen werden viele Dienste des
Systems im Kontext eines eigenen Benutzers betrieben, der in seinen Zugriffsrechten
entsprechend gestaltet werden kann
• "Pseudo-Shells" wie bin/false oder /sbin/nologin stellen keine Umgebung bereit, bzw.
verhindern, dass sich ein Benutzer interaktiv anmelden kann. Das wird als Sicherheitsfeature
verwendet, um Systembenutzer, entsprechend auf das System selbst einzuschränken
• Change Age ( chage ) gibt Informationen aus /etc/shadow lesbarer im Terminal aus und kann
(auch dialoggeführt) Parameter anpassen
• Die Hauptgruppe eines Benutzers ist in /etc/passwd erfasst, daher taucht der Benutzer in
/etc/group nicht noch einmal auf
• Wird eine neue Gruppe erstellt, zählt die GID automatisch bei der zuletzt erstellten ID weiter
hoch
• Das Programm adduser stellt eine dialoggeführte Erstellung von Benutzern inkl. Passwort
bereit
• Das Programm deluser kann mit der Option --remove-all-files den Benutzer mit
sämtlichen Dateien des Benutzers auf dem ganzen System löschen
• Das Verhalten von adduser und deluser kann durch die beiden Dateien
/etc/adduser.conf und /etc/deluser.conf angepasst werden